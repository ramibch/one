

\documentclass[landscape,a4paper, {{ doc_language }}, 10pt]{scrartcl}
\usepackage[utf8]{inputenc}
\usepackage[{{ doc_language }}]{babel}
\usepackage[{{ doc_language }}]{translator}
\usepackage[T1]{fontenc}
\usepackage{tikz}

\usepackage{translator}
\usepackage{fancyhdr}
\usepackage{fix-cm}
\usepackage[hidelinks]{hyperref}
\usepackage{xcolor}


\usepackage[landscape, headheight = 2cm, margin=.5cm,
  top = 3.2cm, nofoot]{geometry}
\usetikzlibrary{calc}
\usetikzlibrary{calendar}
\renewcommand*\familydefault{\sfdefault}
\def\year{ {{- year -}} }


\newcommand{\link}[1]{{\color{blue}\href{#1}{#1}}}
\def\termin#1#2{
    \node [anchor=north west, text width= 3.4cm] at
    ($(cal-#1.north west)+(3em, 0em)$) {\tiny{#2}};
    }
    


\renewcommand{\headrulewidth}{0.0pt}
\setlength{\headheight}{10ex}
\chead{
  \fontsize{60}{70}\selectfont\textbf{\year}
  \Large\textbf{ {{ title }} }\hfill
}

\cfoot{\footnotesize\texttt{  \link{ {{- footer_url -}} } }}

\cfoot{\footnotesize\texttt{  }}


\begin{document}
\pagestyle{fancy}
\begin{center}
\begin{tikzpicture}[every day/.style={anchor = north}]
\calendar[
  dates=\year-01-01 to \year-06-30,
  name=cal,
  day yshift = 3em,
  day code=
  {
    \node[name=\pgfcalendarsuggestedname,every day,shape=rectangle,
    minimum height= .53cm, text width = 4.4cm, draw = gray]{\tikzdaytext};
    \draw (-1.8cm, -.1ex) node[anchor = west]{\footnotesize%
      \pgfcalendarweekdayshortname{\pgfcalendarcurrentweekday}};
  },
  execute before day scope=
  {
    \ifdate{day of month=1}
    {
      \pgftransformxshift{4.8cm}
      \draw (0,0)node [shape=rectangle, minimum height= .53cm,
        text width = 4.4cm, fill = {{ color }}, text= white, draw = gray, text centered]
        {\textbf{\pgfcalendarmonthname{\pgfcalendarcurrentmonth}}};
    }{}
    \ifdate{workday}
    {
      \tikzset{every day/.style={fill=white}}
    }{}
    \ifdate{Saturday}{\tikzset{every day/.style={fill={{color}}!10}}}{}
    \ifdate{Sunday}{\tikzset{every day/.style={fill={{color}}!20}}}{}
    
    \ifdate{equals= {{- holiday.tex_date() -}} }{\tikzset{every day/.style={fill={{holiday_color}}!20}}}{}
    
    },
 execute at begin day scope=
  {
    \pgftransformyshift{-.53*\pgfcalendarcurrentday cm}
  }
];

\termin{ {{- holiday.tex_date() -}} }{ {{- holiday.title -}} }

\end{tikzpicture}
\pagebreak
\begin{tikzpicture}[every day/.style={anchor = north}]
    

\calendar[dates=\year-07-01 to \year-12-31,
  name=cal,
  day yshift = 3em,
  day code=
  {
    \node[name=\pgfcalendarsuggestedname,every day,shape=rectangle, 
      minimum height= .53cm, text width = 4.4cm, draw = gray]{\tikzdaytext};
    \draw (-1.8cm, -.1ex) node[anchor = west]
    {
      \footnotesize\pgfcalendarweekdayshortname{\pgfcalendarcurrentweekday}
    };
  },
  execute before day scope=
  {
    \ifdate{day of month=1} {
    \pgftransformxshift{4.8cm}
    \draw (0,0)node [shape=rectangle, minimum height= .53cm, 
      text width = 4.4cm, fill = {{color}}, text= white, draw = gray, text centered]
    {
      \textbf{\pgfcalendarmonthname{\pgfcalendarcurrentmonth}}
    };
  }{}
  \ifdate{workday}
  {
    \tikzset{every day/.style={fill=white}}
  }{}
  \ifdate{Saturday}{\tikzset{every day/.style={fill={{color}}!10}}}{}
  \ifdate{Sunday}{\tikzset{every day/.style={fill={{color}}!20}}}{}
  
    \ifdate{equals= {{- holiday.tex_date() -}} }{\tikzset{every day/.style={fill={{holiday_color}}!20}}}{}
  
  },
  execute at begin day scope=
  {
    \pgftransformyshift{-.53*\pgfcalendarcurrentday cm}
  }
];


\termin{ {{- holiday.tex_date() -}} }{ {{- holiday.title -}} }

\end{tikzpicture}
\end{center}
\end{document}


