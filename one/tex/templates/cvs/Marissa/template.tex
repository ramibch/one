%%%%%%%%%%%%%%%%%
% This is an sample CV template created using altacv.cls
% (v1.6.5, 3 Nov 2022) written by LianTze Lim (liantze@gmail.com), based on the
% CV created by BusinessInsider at http://www.businessinsider.my/a-sample-resume-for-marissa-mayer-2016-7/?r=US&IR=T
%
%% It may be distributed and/or modified under the
%% conditions of the LaTeX Project Public License, either version 1.3
%% of this license or (at your option) any later version.
%% The latest version of this license is in
%%    http://www.latex-project.org/lppl.txt
%% and version 1.3 or later is part of all distributions of LaTeX
%% version 2003/12/01 or later.
%%%%%%%%%%%%%%%%

%% Use the "normalphoto" option if you want a normal photo instead of cropped to a circle
% \documentclass[10pt,a4paper,normalphoto]{altacv}

\documentclass[10pt,a4paper,ragged2e,withhyper]{altacv-marissa}

%% AltaCV uses the fontawesome5 package.
%% See http://texdoc.net/pkg/fontawesome5 for full list of symbols.

% Change the page layout if you need to
\geometry{left=1.25cm,right=1.25cm,top=1.5cm,bottom=1.5cm,columnsep=1.2cm}

% The paracol package lets you typeset columns of text in parallel
\usepackage{paracol}


% Change the font if you want to, depending on whether
% you're using pdflatex or xelatex/lualatex
\ifxetexorluatex
  % If using xelatex or lualatex:
  \setmainfont{Lato}
\else
  % If using pdflatex:
  \usepackage[default]{lato}
\fi

% Change the colours if you want to
\definecolor{VividPurple}{HTML}{3E0097}
\definecolor{SlateGrey}{HTML}{2E2E2E}
\definecolor{LightGrey}{HTML}{666666}
% \colorlet{name}{black}
% \colorlet{tagline}{PastelRed}
\colorlet{heading}{VividPurple}
\colorlet{headingrule}{VividPurple}
% \colorlet{subheading}{PastelRed}
\colorlet{accent}{VividPurple}
\colorlet{emphasis}{SlateGrey}
\colorlet{body}{LightGrey}

% Change some fonts, if necessary
% \renewcommand{\namefont}{\Huge\rmfamily\bfseries}
% \renewcommand{\personalinfofont}{\footnotesize}
% \renewcommand{\cvsectionfont}{\LARGE\rmfamily\bfseries}
% \renewcommand{\cvsubsectionfont}{\large\bfseries}

% Change the bullets for itemize and rating marker
% for \cvskill if you want to


%% Use (and optionally edit if necessary) this .tex if you
%% want to use an author-year reference style like APA(6)
%% for your publication list
% \input{pubs-authoryear}

%% Use (and optionally edit if necessary) this .tex if you
%% want an originally numerical reference style like IEEE
%% for your publication list


%% sample.bib contains your publications


\begin{document}
\name{ {{- profile.get_tex_value("fullname") -}} }

\tagline{ {{- profile.get_tex_value("jobtitle") -}} }

\tagline{}


%% You can add multiple photos on the left (\photoL) or right (\photoR)


\photoR{2.5cm}{ {{- profile.get_photo_path() -}} }


\personalinfo{
  % You can add your own with \printinfo{symbol}{detail}
  
   \email{ {{- profile.get_tex_value("email") -}} } 
   \phone{ {{- profile.get_tex_value("phone") -}} } 
   \location{ {{- profile.get_tex_value("location") -}} } 
   \homepage{ {{- profile.get_tex_value("website") -}} } 

  % \mailaddress{}
  % \twitter{@marissamayer}
  % \linkedin{marissamayer}
  % \github{github.com/mmayer} % I'm just making this up though.
  % \orcid{0000-0000-0000-0000} % Obviously making this up too.
  %% You can add your own arbitrary detail with
  %% \printinfo{symbol}{detail}[optional hyperlink prefix]
  % \printinfo{\faPaw}{Hey ho!}
  %% Or you can declare your own field with
  %% \NewInfoFiled{fieldname}{symbol}[optional hyperlink prefix] and use it:
  % \NewInfoField{gitlab}{\faGitlab}[https://gitlab.com/]
  % \gitlab{your_id}
	%%
  %% For services and platforms like Mastodon where there isn't a
  %% straightforward relation between the user ID/nickname and the hyperlink,
  %% you can use \printinfo directly e.g.
  % \printinfo{\faMastodon}{@username@instace}[https://instance.url/@username]
  %% But if you absolutely want to create new dedicated info fields for
  %% such platforms, then use \NewInfoField* with a star:
  % \NewInfoField*{mastodon}{\faMastodon}
  %% then you can use \mastodon, with TWO arguments where the 2nd argument is
  %% the full hyperlink.
  % \mastodon{@username@instance}{https://instance.url/@username}
}

\makecvheader

%% Depending on your tastes, you may want to make fonts of itemize environments slightly smaller
\AtBeginEnvironment{itemize}{\small}

%% Set the left/right column width ratio to 6:4.
\columnratio{0.6}

% Start a 2-column paracol. Both the left and right columns will automatically
% break across pages if things get too long.
\begin{paracol}{2}

% Experience

\cvsection{ {{- profile.get_tex_value("experience_label") -}} }

  \cvevent{ {{- experience.get_tex_value("title") -}} }{ {{- experience.get_tex_value("company") -}} }{ {{ experience.get_tex_value("start_date") }} -- {{ experience.get_tex_value("end_date") }} }{}
  {{- experience.get_tex_value("description") | linebreaks -}} \\
  \divider



% Education

\cvsection{ {{- profile.get_tex_value("education_label") -}} }

  \cvevent{ {{- education.get_tex_value("title") -}} }{ {{- education.get_tex_value("institution") -}} }{ {{ education.get_tex_value("start_date") }} -- {{ education.get_tex_value("end_date") }} }{}
  {{- education.get_tex_value("description") | linebreaks -}} \\
  \divider



% Projects

\cvsection{ {{- profile.get_tex_value("project_label") -}} }
  
  \cvevent{ {{- project.get_tex_value("title") -}}  }{ {{- project.get_tex_value("organization") -}} }{}{}
  {{- project.get_tex_value("role") -}}\\
  \divider
  


% Publications

\cvsection{ {{- profile.get_tex_value("publication_label") -}} }
  
  \cvevent{ {{- publication.get_tex_value("title") -}} }{ {{- publication.get_tex_value("publisher") -}} }{ {{- publication.get_tex_value("date") -}} }{}
  {{- publication.get_tex_value("authors") -}} \\
  \divider
  


\newpage



%% Switch to the right column. This will now automatically move to the second
%% page if the content is too long.
\switchcolumn

% about

\cvsection{ {{- profile.get_tex_value("about_label") -}} }
\begin{quote}
  {{- profile.get_tex_value("about") | linebreaks -}}
\end{quote}



% achievements

\cvsection{ {{- profile.get_tex_value("achievement_label") -}} }

  \cvachievement{\faTrophy}{ {{- achievement.get_tex_value("date") -}} }{ {{- achievement.get_tex_value("title") -}} }
  \divider



% Skills

\cvsection{ {{- profile.get_tex_value("skill_label") -}} }

  \cvskill{ {{- skill.get_tex_value("name") -}} }{ {{- skill.level_base_5_float -}} }



% Languages

\cvsection{ {{- profile.get_tex_value("language_label") -}} }

  \cvskill{ {{- language.get_tex_value("name") -}} }{ {{- language.level_base_5_float -}} }



\end{paracol}

\end{document}
